\newcommand{\basedir}{./fablab-document/}
\documentclass{\basedir/vsh-document}

\usepackage{tabularx} % Tabellen mit bestimmtem Breitenverhältnis der Spalten

\newcommand{\thickhline}{\noalign{\hrule height 2pt}}
\usepackage{eurosym}
\renewcommand{\texteuro}{\euro}

\title{Wartung 3D-Drucker}
\author{Stefan Kaufmann, nach Vorlage von Patrick Kanzler}

\fancyfoot[L]{Wartungsliste 3D-Drucker}
\fancyfoot[C]{}
%\fancyfoot[R]{Liste Nr. \underline{~~~~~~~~}}

\begin{document}
Bitte hier alle Wartungen und Prüfungen dokumentieren

Zu tun: \emph{Monatlich, Drucker ausgeschaltet:} Build Plate mit Mikrofasertuch unter Wasser reinigen; Staub aus Druckraum entfernen

\emph{Dreimonatlich, Drucker ausgeschaltet:} Achsen/Gleitlager oelen, vorgeschrieben: Unilube. Je ein Tropfen auf X-, Y- und beide Z-Achsen, Gleitlager mehrfach hin- und herbewegen, Schmutzreste an den Enden der Lagerstangen ggf. vorher entfernen. Achsspiel prüfen. Spannung der kurzen Zahnriemen prüfen: Riemen sollten sich nicht aufeinander drücken lassen; Spannung mehr wie bei einer Gitarrensaite. Lüfter des Druckkopfs auf Schmutz prüfen, ggf reinigen. Gewindestange der Z-Achse neu schmieren, ggf. vorher Schmutz entfernen, vorgeschrieben: MAGNALUBE. Ultimaker 3: Silikon-Abdeckung der Düsen auf Verschleiß prüfen.

\emph{Jährlich:} Filament entfernen, Drucker ausschalten. Filamentzuführung mit Druckluft reinigen. Bowdenzüge ausbauen, mit Druckluft reinigen, ggf. tauschen.

\newcommand{\bsp}[1]{\textcolor{gray}{\itshape #1}}
\newcommand{\beispielzeile}[5]{\bsp{#2} & \bsp{#3} & \bsp{#4} \\ \hline}
\newcommand{\leerzeile}{\vbox{\vspace{2.4em}} & & \\ \hline}
\vspace{-.4em}
\begin{tabularx}{\textwidth}{|c|X|c|} \hline
\bfseries Datum      &  \bfseries Tätigkeit/Anlass  & \bfseries Name \\\thickhline
\beispielzeile{BSP}{12.5.2018}{ Dreimonatliche Wartung komplett, alles i.O. }{Name, Handzeichen}
\leerzeile
\leerzeile
\leerzeile
\leerzeile
\leerzeile
\leerzeile
\leerzeile
\leerzeile
\leerzeile
\leerzeile
\leerzeile
\leerzeile
\leerzeile
\leerzeile
\leerzeile
\leerzeile
\end{tabularx}



\end{document}
