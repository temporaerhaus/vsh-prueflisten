\newcommand{\basedir}{./fablab-document/}
\documentclass{\basedir/tph-document}

\usepackage{tabularx} % Tabellen mit bestimmtem Breitenverhältnis der Spalten

\newcommand{\thickhline}{\noalign{\hrule height 2pt}}
\usepackage{eurosym}
\renewcommand{\texteuro}{\euro}

\title{Wartung Kaffeeautomat}
\author{Stefan Kaufmann, nach Vorlage von Patrick Kanzler}

\fancyfoot[L]{Wartungsliste Kaffeeautomat}
\fancyfoot[C]{}
%\fancyfoot[R]{Liste Nr. \underline{~~~~~~~~}}

\begin{document}
Bitte hier alle nicht-täglichen Reinigungen und Tausch der Filterpatronen eintragen.

Zu tun: Tägliche Reinigung: Schale leeren, Spültasse leeren, Front mit sauberem Schwammtuch abwischen

Wenn Maschine „Reinigung“ verlangt: Nach Handbuch vorgehen, Reinigungstabletten befinden sich im Kaffeezubehör-Schrank. Reinigung in der Liste vermerken.

Alle 2–3 Monate: Filterpatrone erneuern.

\newcommand{\bsp}[1]{\textcolor{gray}{\itshape #1}}
\newcommand{\beispielzeile}[5]{\bsp{#2} & \bsp{#3} & \bsp{#4} \\ \hline}
\newcommand{\leerzeile}{\vbox{\vspace{2.4em}} & & \\ \hline}
\vspace{-.4em}
\begin{tabularx}{\textwidth}{|c|X|c|} \hline
\bfseries Datum      &  \bfseries Tätigkeit/Anlass  & \bfseries Name \\\thickhline
\beispielzeile{BSP}{12.5.2018}{ Reinigungszyklus, Filter getauscht }{Name, Handzeichen}
\leerzeile
\leerzeile
\leerzeile
\leerzeile
\leerzeile
\leerzeile
\leerzeile
\leerzeile
\leerzeile
\leerzeile
\leerzeile
\leerzeile
\leerzeile
\leerzeile
\leerzeile
\leerzeile
\leerzeile
\leerzeile
\end{tabularx}



\end{document}
